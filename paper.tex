\documentclass[10pt,letterpaper]{article}
\usepackage[top=0.85in,left=2.75in,footskip=0.75in]{geometry}

% amsmath and amssymb packages, useful for mathematical formulas and symbols
\usepackage{amsmath,amssymb}

% Use adjustwidth environment to exceed column width (see example table in text)
\usepackage{changepage}

% Use Unicode characters when possible
\usepackage[utf8x]{inputenc}

% textcomp package and marvosym package for additional characters
\usepackage{textcomp,marvosym}

% cite package, to clean up citations in the main text. Do not remove.
\usepackage{cite}

% Use nameref to cite supporting information files (see Supporting Information section for more info)
\usepackage{nameref,hyperref}

% line numbers
\usepackage[right]{lineno}

% ligatures disabled
\usepackage{microtype}
\DisableLigatures[f]{encoding = *, family = * }

% color can be used to apply background shading to table cells only
\usepackage[table]{xcolor}

% array package and thick rules for tables
\usepackage{array}

% enumerate package lets us use letters instead of numbers
\usepackage{enumerate}

% create "+" rule type for thick vertical lines
\newcolumntype{+}{!{\vrule width 2pt}}

% create \thickcline for thick horizontal lines of variable length
\newlength\savedwidth
\newcommand\thickcline[1]{%
  \noalign{\global\savedwidth\arrayrulewidth\global\arrayrulewidth 2pt}%
  \cline{#1}%
  \noalign{\vskip\arrayrulewidth}%
  \noalign{\global\arrayrulewidth\savedwidth}%
}

% \thickhline command for thick horizontal lines that span the table
\newcommand\thickhline{\noalign{\global\savedwidth\arrayrulewidth\global\arrayrulewidth 2pt}%
\hline
\noalign{\global\arrayrulewidth\savedwidth}}

\usepackage{color}

% Remove comment for double spacing
%\usepackage{setspace}
%\doublespacing

% Text layout
\raggedright
\setlength{\parindent}{0.5cm}
\textwidth 5.25in
\textheight 8.75in

% Bold the 'Figure #' in the caption and separate it from the title/caption with a period
% Captions will be left justified
\usepackage[aboveskip=1pt,labelfont=bf,labelsep=period,justification=raggedright,singlelinecheck=off]{caption}
\renewcommand{\figurename}{Fig}

% Use the PLoS provided BiBTeX style
\bibliographystyle{plos2015}

% Remove brackets from numbering in List of References
\makeatletter
\renewcommand{\@biblabel}[1]{\quad#1.}
\makeatother

% Leave date blank
\date{}

% Header and Footer with logo
\usepackage{lastpage,fancyhdr,graphicx}
\usepackage{epstopdf}
\pagestyle{myheadings}
\pagestyle{fancy}
\fancyhf{}
\setlength{\headheight}{27.023pt}
\lhead{\includegraphics[width=2.0in]{PLOS-submission.eps}}
\rfoot{\thepage/\pageref{LastPage}}
\renewcommand{\footrule}{\hrule height 2pt \vspace{2mm}}
\fancyheadoffset[L]{2.25in}
\fancyfootoffset[L]{2.25in}
\lfoot{\sf PLOS}

%% Include all macros below
\newcommand{\fixme}[2]{\textsc{\textbf{{#1}: {#2}}}}
\newcommand{\recommend}[1]{\textit{#1}}
\newcommand{\withurl}[2]{{#1}\footnote{\texttt{#2}}}
\newcommand{\rulemajor}[1]{\section{#1}}
\begin{document}
\vspace*{0.2in}

\begin{flushleft}
{\Large
\textbf\newline{Ten Simple Rules for Collaborative Lesson Development}
}
\newline
\\
{Gabriel~A.~Devenyi}\textsuperscript{1},
{R\'{e}mi~Emonet}\textsuperscript{2},
{Rayna~M.~Harris}\textsuperscript{3},
{Kate~Hertweck}\textsuperscript{4},
{Damien~Irving}\textsuperscript{5},
{Ian~Milligan}\textsuperscript{6},
{Greg~Wilson}\textsuperscript{7, *}
\\
\textbf{1} Douglas Mental Health University Institute / gdevenyi@gmail.com \\
\textbf{2} Universit\'{e} Jean Monnet / remi.emonet@univ-st-etienne.fr \\
\textbf{3} University of Texas at Austin / rayna.harris@utexas.edu \\
\textbf{4} University of Texas at Tyler / khertweck@uttyler.edu \\
\textbf{5} CSIRO Oceans and Atmosphere / irving.damien@gmail.com \\
\textbf{6} University of Waterloo / i2millig@uwaterloo.ca \\
\textbf{7} Rangle.io / gvwilson@third-bit.com
\\
* Corresponding author.
\end{flushleft}

\section*{Abstract}

FIXME: abstract

\section*{Author Summary}

FIXME: author summary

\section*{Introduction}

Thousands of university lecturers teach subjects ranging from first year biology,
to graduate-level courses in Indian film.
Some might use a common textbook written by a single author or two,
but other than that they usually develop and improve their course materials in isolation.
It is staggering to think how many wheels are being re-invented,
how the progress of teaching in these fields has been held back by
the lack of community collaboration and sharing,
and how much valuable time has been wasted.

This problem extends beyond university teaching,
but it is curious that it is endemic in tertiary education,
given that research depends so critically on collaboration and sharing,
and that most researchers complain about how much time teaching takes away from research.

The software community's model of collaborative code development is an alternative.
The authors first encountered it in the context of
lessons on basic lab skills for research computing
in the sciences and humanities...

FIXME: fill in introduction

\rulemajor{Clarify your audience.}

The first requirement for building lessons together is
to know who they are being built for.
``Archaeology students'' is far too vague:
are you and your collaborators thinking of
first-year students who need an introduction to the field,
graduate students who intend to specialize in the sub-discipline which is the lesson's focus,
or someone in between?
Prerequisite knowledge,
equipment or software required,
how much time learners will actually have:
if different contributors believe different things about these,
they will find it difficult or impossible to work together.

Thinking systematically about difficulty levels can help manage expectations.
For example,
the Programming Historian project labels lessons ``beginner''",
``intermediate'',
and ``advanced''
to help authors write at appropriate levels.
However,
this only works if there is prior agreement on what those terms mean.

Rather than itemizing prior knowledge and learning objectives,
it can be helpful to write \emph{learner profiles}.
This technique is borrowed from user interface design
to help authors think about their audience's needs
and give them a shorthand for talking about specific cases.

Learner profiles have five parts:
the learner's general background,
what they already know,
what *they* think they want to do,
how the material will help them,
and any special needs they might have.
A learner profile for a weekend programming workshop for new college students might be:

\begin{enumerate}

\item
  Jorge has just moved from Costa Rica to Canada
  to study agricultural engineering.

\item
  Other than using Excel, Word, and the Internet,
  Jorge's most significant previous experience with computers is
  helping his sister build a WordPress site for the family business.

\item
  Jorge needs to measure properties of soil from nearby farms
  using a handheld device that sends text files to his computer.
  Right now, Jorge has to open each file in Excel,
  crop the first and last points,
  and calculate an average.

\item
  This workshop will show Jorge how to write a little Python program
  to read the data,
  select the right values from each file,
  and calculate the required statistics.

\item
  Jorge can read English proficiently,
  but still struggles sometimes to keep up with spoken conversation
  (especially if it involves a lot of new jargon).

\end{enumerate}

\rulemajor{Build community around lessons.}

Lessons don't maintain themselves. 
Unlike traditional academic work, 
which may have shelf time measured 
in years (if not longer), 
technical lessons do not maintain themselves. 
Versions change, dependencies break, 
and what was cutting edge in 2017 may be dated 
and less useful in 2018. 
This needs to be clear to contributors, 
so they understand that this lesson is just 
the starting point. 
Sustainability needs to be front of mind.

The focus accordingly needs to be on creating a community. 
Authors cannot be expected to maintain continual vigilance on a lesson,
but this is necessary if one expects continual use of a lesson! 
By working online, creating opportunities for collaboration 
and contribution, many eyes can keep the lesson usable. 
As we note below, 
this only works if you make space for 
little contributions just as you do for larger ones.

\rulemajor{Reduce, re-use, recycle.}

Don't create a new lesson if there is an existing one that you can use or contribute to.
Just as a scholar would not write a paper without a literature review,
the same holds for lessons.
Before writing that introduction to the Bash command line,
for example,
do a search:
has anybody else written it?
Is it complementary to your goals?
Could it be tweaked or modified to meet your own goals?
Could your planned lesson be tweaked to compliment the existing lessons so that topics aren't duplicated? 

FIXME: This is a good place to talk about discoverability as well as licensing

Before re-using content,
make sure to check that the lesson is licensed in an open manner.
Both Programming Historian and Software Carpentry license their material under
the Creative Commons - Attribution license
(see https://creativecommons.org/licenses/by/2.0/).
This allows people to share and adapt material for any purpose,
even commercial ones,
without asking permissions as long as they continue to share the content in a similar manner.

The same questions of re-use come when thinking about recycling content of a lesson,
such as images, data, figure, or code.
Does the license cover that as well?
If not,
then ask permission,
just as you would for any other material.

As you collaboratively develop your own lessons,
make sure to think about the collaborative contexts of the entire field.

\rulemajor{Build modular lessons that can be re-purposed.}

Every instructor's needs are different,
so build small chunks that can be re-purposed in many ways. 
A university lecturer in meteorology, for instance,
might construct a course by bringing together lessons on differential equations,
fluid mechanics and absorption spectroscopy. 
This task is greatly simplified if existing courses on mathematics, 
physics and chemistry consist of numerous small, discrete lessons,
as opposed to a few large, monolithic lessons.

Note that doing this shifts the instructor's burden from writing to finding and synthesizing.
Both of which are easier if lessons have been designed by people with a shared world-view (rule 6),
and if lessons clearly signal what they cover (Rule~1).
In particular,
if lessons reference specific points in the model curriculum guidelines promulgated by many professional societies,
it can be much easier for people to find them and integrate them.

Note also that individual lesson topics hinge on the learners having the proper prerequisite background,
so this rule further emphasizes the importance of rule 1.

\rulemajor{Encourage and empower contributors.}

It takes a village to ensure
the steady improvement and adaptation that will give a lesson a long life.
Ensuring this, however, is not straightforward.

Making the process explicit is key.
New contributors require a straightforward and transparent introduction
to understand the process of tweaking and adapting lessons.
Licensing, code of conduct, governance, and contribution procedure
all need to be explicit rather than implicit
to lower the social barriers to contribution.

Tools can help,
especially if they support pre-merge review.
Yet some of these tools come with a considerable up-front learning curve.
GitHub with pull requests may be ideal,
but allowing people to edit a Google Doc,
or to facilitate Wiki-based editing,
can help get conversations started.
In a pinch,
work with contributors to find a comfort zone:
don't let a collaboration flounder on technological choice.
But make sure not to overwhelm either:
threaded discussion (or GitHub issues) will help increase the signal-to-noise ratio.

Ultimately,
community dynamics are more important than platform.
We have seen the 90/9/1 rule:
90\% of people watch, 9\% discuss, 1\% do.
This requires a gentle on-ramp for new contributors,
that we make it easy for people to submit errata and suggestions,
and that editors may need to do triage to ensure that voices are heard.
Finally, with so many voices and contributions come caution.
Working in the open can be great,
but can also unintentionally suppress voices.
At Programming Historian,
for example,
an ombudsperson is available for private chats and facilitation.

\rulemajor{Teach best practices for lesson development.}

Decades of pedagogical research has yielded many insights into
how best to build and deliver lessons \cite{hlw}.
Unfortunately,
since most college and university faculty have little or no training in education,
this knowledge and expertise is rarely transferred into classroom practice.

Experience shows that even a brief introduction to a handful of key practices
can help collaborative lesson development in at least three ways.
If people have a shared understanding of how lessons should be developed,
it is easier for them to work together.
Less obviously,
if people have a shared model of how lessons are going to be *used*,
they are more likely to try to build the same kind of material.
Finally,
teaching people how to teach is a great way to introduce them to each other and build community.

An example of a particular lesson development practice is reverse instructional design
\cite{wiggins-mctighe}.
When this is used,
lessons are built by:

\begin{itemize}

\item
  identifying the learning objectives,

\item
  creating \emph{summative assessments} to determine whether those objectives have been met,

\item
  designing \emph{formative assessments} to gauge learners' progress
  and give them a chance to practice key skills,

\item
  putting those formative assessments in order,
  and

\item
  writing lessons to connect each to the next.

\end{itemize}

This methodology is effective in its own right,
but the real benefit comes when everyone understands what is supposed to happen
in what order.

Another example is Software Carpentry's instructor training program.
First offered in 2012,
this gradually turned into a two-day course delivered both in-person and online
\cite{lessons-learned}.
This crash course was a natural complement to Software Carpentry's regular workshops:
while the latter were created so that researchers wouldn't have to teach themselves how to program,
the former ensures that they don't have to teach themselves how to teach.
Instructor training now serves and an onboarding process to get everyone on the same page
regarding who lessons are for,
how they are delivered,
and how they are maintained.

\rulemajor{Publish periodically and recognize contributions.}

Similar to publishing software,
lessons should have releases of fixed content
so that learners or instructors who may wish to use the material have a stable version to refer to
for the duration of their use.
These releases should be periodic
so that improvements and adjustments are made available
for new learners and instructors just starting their use.
Periodic releases are also essential
for enabling recognition of the contribution of authors and maintainers.

The traditional academic system has limited ways of recognizing contributions.
Until systems have been expanded to improve this,
it is important to publish your lessons using a mechanism that provides recognition to the contributors.
Currently an effective way to do this is
to publish lessons with a DOI supplied by (FIXME list possibilities).
Contributors can be listed as authors
and the maintainers of the lesson as editors
to provide differentiated recognition of their contributions.

A lesson release is a good opportunity to bring the material into a stable shape
by fixing outstanding issues and merging contribution.
The complete name and possible identifier like ORCID should also be gathered for every contributor.
Version control helps a lot in continuously maintaining a list of contributors
but also in remembering which version is used for release
(e.g., using branches or tags).
A lesson release is here to stay
and it is recommended to use a consistent naming scheme from the beginning.
A convenient naming scheme is to use the full year and month of the release,
like ``2017.05'',
as it conveys a clear notion of date,
even to people that are new to the project.

Once the lesson is ready for release and has a name,
it should be built
and submitted to DOI supplier with its metadata,
such as lesson name, authors (e.g., contributors) and editors (e.g., maintainers).
Due to the collaborative aspect of the lesson development,
the number of authors and even lessons can grow very fast
and it is worth spending some time on automating this release process.
After the formal publishing,
it is also a good idea to publish the built lesson for direct access by the community,
making it clear which version is displayed and what is its corresponding DOI. 
 
\rulemajor{Evaluate lessons at several scales.}

The purpose of feedback is to guide lesson development
so that authors aren't designing and arguing in a vacuum.
What people immersed in the lessons think needs fixing
can often differ from what learners think.

Micro-scale feedback can be gathered by an instructor while teaching a particular lesson.
Learners might provide feedback on things like typographical errors,
clarity/ease of quiz questions and/or the order in which topics are presented,
which the instructor can enter into a work-tracking system (e.g., GitHub issues) at the end of class.
As well as encouraging direct verbal feedback,
it's a good idea to provide learners with a means to provide feedback anonymously during class 
(e.g., on small pieces of paper like sticky notes).

Pre- and post-class surveys can be used to discover larger macro-scale issues.
These issues often relate to the fact that it can be difficult for lesson developers
to fully understand the frame of reference of their audience.
For instance, a lesson might inadvertently assume prior knowledge that many learners don't have,
which is information that can be collected easily in a survey.
If possible, it's a good idea to conduct the post-class survey 30-60 days after the fact.
This allows people time to reflect,
meaning they are more likely to give accurate feedback on what they learned
rather than how entertained they were.

Pre- and post-class surveys also are essential in focusing in on the audience for a given lesson.
It can sometimes be difficult for lesson developers to understand the frame of a given learner
so surveys (particularly post-class surveys) can reveal hidden assumed knowledge
that can be expanded on or acknowledged.

\rulemajor{Point to alternate resources.}

Most learners are unlikely to completely absorb the topic you are trying to
present using only your lesson.
There are many angles and approaches for any given topic.
Make an attempt to find these alternative resources and offer them at strategic points in the lesson,
and collect them at the end.
Some learners prefer different material styles, so seek out textbooks, technical
documentation, videos and web pages which may present the material in alternative styles.
If a community or discussion forum exists for the topic, be sure to include these resources.

Beyond the topic of your lesson, learners can go in many directions after completion.
Provider learners with extensions and expansions of the topic from a variety of
applications and viewpoints.

\begin{itemize}

\item
  The lesson is a beginning, not an end, for the learner

\item
  point them at similar material

\item
  and where to go next when they want to expand their knowledge.

\item
  A lesson is one piece of a larger engagement your learners are having
  with a topic. Enable their further engagement.

\end{itemize}

\rulemajor{You can't please everyone.}

No single lesson can be right for every learner.
Two people with no prior knowledge of a specific subject
may still be able to move at very different speeds
because of different levels of general background knowledge.
Similarly,
lessons on ecology for learners in Utah and Florida
will probably be more relatable if they use different examples.
Equally,
no lesson development community can serve all purposes.
Some groups may prioritize rapid evolution,
while others may prefer a ``measure twice, cut once'' approach.

If there are several complementary ways to explain something,
or several points of views that can cohabit respectfully,
it may be possible to present them side by side.
There are actually good pedagogical reasons to do this even if contributors *don't* disagree:
weighing alternatives fosters higher-order thinking,
and as Caulfield has written,
the most popular online question and answer sites
are successful in part because they present a chorus of explanations
geared at different levels and needs
rather than a single one (FIXME: cite choral explanations).

But sometimes choices must be made.
The open source software community has wrestled with these issues for three decades,
and has evolved some best practices to handle them
\cite{producing-oss}.
As discussed in Rule~5,
the first step is to have a clear governance structure and a clear, permissive license.
Minor disagreements should be discussed openly and respectfully.
If they cannot be resolved---i.e., if they turn out not to be so minor after all---then
contributors should split off and evolve the lesson in the way they see best.
(This is one of the reasons to have a permissive license.)

Forking rarely happens in practice.
When it does,
it is important to remember that we all share the same vision of better lessons, built together.

\section*{Conclusion}

FIXME

\bibliography{paper}

\end{document}

